\chapter*{Abstract} %l'asterisco dopo chapter serve per visualizzare il capitolo come "non numerato"
Questa tesi si propone di analizzare e sperimentare algoritmi di deep learning sviluppati nel campo della novel view synthesis e dei Neural Radiance Fields (NeRF), per affrontare la complessa sfida della ricostruzione di ambienti 3D. In particolare, ci si è focalizzati sugli algoritmi basati sul metodo del 3D Gaussian Splatting, uno dei più recenti e avanzati approcci all’avanguardia nel settore. Questi algoritmi sono stati testati su una varietà di dataset, con l’obiettivo di costruire un benchmark solido e di confrontare le loro prestazioni con quelle di altri metodi di riferimento disponibili nella letteratura. Per garantire un’analisi completa e affidabile, la valutazione delle prestazioni è stata condotta utilizzando sia metriche qualitative che quantitative. In particolare, sono stati impiegati questionari di valutazione qualitativa,  metriche senza riferimento per le immagini e metriche di distanza per la valutazione delle mesh 3D ricostruite.
