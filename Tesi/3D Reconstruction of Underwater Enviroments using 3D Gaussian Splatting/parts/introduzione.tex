\chapter{Introduzione}
La ricostruzione di ambienti tridimensionali (3D) rappresenta un compito intrinsecamente complesso, che storicamente è stato affrontato prevalentemente attraverso l’uso di tecniche di fotogrammetria. Questi approcci si basano sull’elaborazione di immagini bidimensionali (2D) per ricreare la geometria di una scena o di un oggetto in tre dimensioni, un processo che, pur risultando efficace, richiede tempi di calcolo piuttosto elevati e risorse computazionali significative. Inoltre, la precisione della ricostruzione è spesso legata alla qualità e al numero delle immagini disponibili, nonché alla capacità del software di gestire complessità geometriche e texture variabili.
\\\\
Negli ultimi anni, tuttavia, grazie ai progressi nel campo del deep learning e dell’intelligenza artificiale, sono emerse nuove tecniche che offrono un’alternativa promettente ai metodi tradizionali. In particolare, algoritmi avanzati basati su reti neurali stanno rivoluzionando il settore della ricostruzione 3D, permettendo la generazione di viste sintetiche (novel view synthesis) con una velocità e una precisione prima impensabili.
\\\\
In computer grafica, la sintesi di viste (view synthesis), o sintesi di viste nuove (novel view synthesis), è un compito che consiste nel generare immagini di un soggetto, o di una scena specifica, da un determinato punto di vista, avendo come uniche informazioni disponibili più o meno fotografie scattate da diversi punti di vista.
Questo task è stato affrontato con successo solo di recente (tra la fine degli anni 2010 e l’inizio degli anni 2020), principalmente grazie ai progressi nel machine learning. Metodi di successo degni di nota, di cui discuteremo in questa tesi, sono i Neural Radiance Fields (NeRF) e il 3D Gaussian Splatting.
\\\\
Come si può facilmente intuire, questo compito è già di per sé altamente complesso, ma la difficoltà cresce in modo esponenziale quando le scene da ricostruire si trovano in ambienti subacquei. \\
Le condizioni uniche del mondo sottomarino introducono sfide aggiuntive, che rendono il processo di ricostruzione 3D ancora più impegnativo. Questo aumento di complessità è dovuto principalmente a due fattori fondamentali:

	•	La presenza dell’acqua influisce notevolmente sulla propagazione della luce, causando fenomeni come rifrazione, diffusione e assorbimento. Questi effetti ottici possono distorcere le immagini catturate, rendendo difficile una ricostruzione accurata della scena. Inoltre, la luminosità in ambienti sottomarini varia in base alla profondità, alla torbidità dell’acqua e alla presenza di particelle sospese, introducendo ulteriori incertezze nella qualità delle immagini.
 
	•	Fotografare in ambienti sottomarini è un’operazione complessa che richiede attrezzature specializzate e condizioni ambientali favorevoli. Di conseguenza, raccogliere un numero sufficiente di immagini ad alta qualità per la ricostruzione 3D è spesso complicato. La scarsa reperibilità di dataset ampi e diversificati costituisce un ostacolo significativo per l’addestramento di algoritmi di deep learning, che richiedono grandi quantità di dati per ottenere risultati precisi.
\\\\
 Affrontare questa sfida complessa, identificando gli algoritmi più adatti per la ricostruzione di scene subacquee, offre notevoli vantaggi a diversi settori. Ad esempio, la ricostruzione 3D di fondali marini può aprire la strada a esperienze immersive, consentendo la visita di ambienti sottomarini attraverso visori per la realtà aumentata (AR) o virtuale (VR), senza la necessità di immergersi fisicamente nell’oceano. Questa applicazione non solo rende l’esplorazione di tali luoghi più accessibile, ma offre anche opportunità educative e di ricerca, permettendo a studenti e ricercatori di studiare ecosistemi marini da una prospettiva unica.

Inoltre, le ricostruzioni ottenute mediante questi algoritmi possono trovare impiego significativo nel settore dell’entertainment, in particolare nella realizzazione di film e videogiochi. Grazie alla capacità di generare ambienti tridimensionali realistici, i produttori possono risparmiare un notevole quantitativo di tempo e risorse che altrimenti sarebbe necessario investire in modellazione 3D tradizionale. Questo non solo accelera il processo di produzione, ma consente anche agli artisti di concentrarsi su aspetti creativi e narrativi, piuttosto che su dettagli tecnici, migliorando così la qualità complessiva dei contenuti.